\renewcommand{\abstractname}{Sammanfattning}
\begin{abstract}


Telenors svenska n�tverks�vervakning har utvecklat ett system f�r att
automatiskt generera n�tverkskartor i formatet SVG.
De har st�llt fr�gan om det g�r att g�ra dessa interaktiva och koppla
ihop dem med befintliga verktygsprogram.

Denna studie visar tekniker som kan anv�ndas f�r att utveckla ett system
som g�r SVG-baserade n�tverkskartor interaktiva i en webbl�sare.

Ett system baserat p� �ppen mjukvara och �ppna standarder utvecklas f�r att
visa hur teknikerna kan anv�ndas i praktiken.
Systemets arkitektur beskrivs i tre systemvyer.
N�tverkskartorna berikas med bindningar mellan h�ndelser i
webbl�saren och JavaScript-funktioner genom att transformera dem med XSLT.
Anv�ndargr�nssnittet utg�rs av SVG-objekt och JavaScript varifr�n det
g�r att asynkront anropa program p� en webbserver.

Studien avslutas med att utv�rdera systemet och ge f�rslag p� hur det
kan f�rb�ttras.

\end{abstract}
